\section{Introduction}

This Bachelor's thesis deals with the formalisation, in the Isabelle/HOL theorem prover, of parts of
the book \emph{Types and Programming Languages} \cite{pierce-2002-TAPL}, hereafter abbreviated
\emph{TAPL}, by Benjamin~C.~Pierce. This work concentrate on four of the languages, ranging from
simple arithmetic expressions to the fully fledged $\lambda$-calculus, present in the first two
sections, namely ``Untyped Systems'' and ``Simple Types''.

The main motivation to have chosen this subject is the intersection of personal interest and of
opportunities provided by my intership at the chair for logic and verification at Technische
Universität München. Having gradually developped an interest for programming languages in the last
years, I was eager to learn more about the theory behind type systems. Pierce's book stood out as
a reference recommended for a deep introduction to the main elements of this field. Also, as part of
my intership, I worked on the implementation of the (co)datatype module in the Isabelle/HOL
theorem proover. Having experienced the implementator role, I also wanted to learn about the user
role and the process of formalization. Thus, the choice of this subject for this thesis was an
opportunity to fulfill both goals.

Before digging into the realm of formalizations, we first introduce the required background (Section
\ref{sec:background}) in $\lambda$-calculus, type systems and Isabelle/HOL. The $\lambda$-calculus
is a core calculus that captures the essentials features of programming languages. That is, there
exists a way to encode high level features such as recursion, datatypes, records, etc. Such calculus
can come, as programming languages do, in two variants: typed and untyped. A type system is a
syntactic method for automatically checking the absence of certain erroneous behaviors. To formalize
these we used Isabelle/HOL, an interactive theorem prover based on higher order logic. It resembles
a programming language in that one can define datatypes and functions. The difference is that it
allows to postulate properties of the formerly defined elements and to provide machine checked poofs
that those properties are theorems.

The formalizations we perform all have a direct corespondance with chapters from \emph{TAPL}
(Section \ref{sec:structure-of-formalization}). We provide one Isabelle/HOL theory file per chapter.

The untyped arithmetic expressions language (Section \ref{sec:untyped-arith-expr}) serves as a
warm-up to experiment with the general structure of formalizations. It consists of boolean
expressions and natural numbers. This simplicity allows us to concentrate on the translation to
Isabelle/HOL of the definitions found in the book and to accustom ourselves with the notation. Most
of our definitions and theorems closely follows the ones from the book. The main exceptions are
that we expose some hypothesis that are implicit in the book and that we use a different definition
of the mutli-step evaluation relation.

The formalization (Section \ref{sec:nameless-rep-of-terms}) of the nameless representation of terms,
also known as ``de Bruijn indices'', was not initially planned but arose from the need to use a
concreate representation for variables in the $\lambda$-calculus. Our formalization closely follows
the book.

While the previous formalizations are either a warm-up or a representation necessity, the untyped
$\lambda$-calculus (Section \ref{sec:untyped-lambda-calculus}) is the first core calculus we
formalized. We differ from the book significantly in the evaluation relation. In \emph{TAPL}, the
evaluation relation assumes that name clashes in variables are automatically solved by renaming
them and, thus, ignore this possibility from there on. Such an assumption is not accepted by
computer-verified proof. We use ``de Bruijn indices'' as representation for variable to encode this
assumption. Also, since the chapter in the book is more focused on explaining the
$\lambda$-calculus, it contains no meaningful theorems. Nevertheless, we revisit the properties
introduced with the arithmetic expressions language and either prove that they are still theorems
or disprove them.

The typed arithmetic expressions language (Section \ref{sec:typed-arith-expr}) is again a warm-up;
this time, to experiment with the formalization of a type system. Our formalization closely follows
the book.

The simply typed $\lambda$-calculus (Section \ref{sec:simply-typed-lambda-calculus}) is the second
core calculus we formalize. Here, we differ significantly from the book, mainly because of our use
of ``de Bruijn indices'' but also because of our reprensentation of the typing context, we need to
adapt some lemmas and replace others. This is certainly the most challenging part of the
formalization, since we cannot follow the described proofs anymore and must find the right
assumptions for our lemmas. In spite of these differences, our formalization still respects the
spirit of the book since only the helper lemmas change, and the important theorems remain the same.

All sections combined, the formalization consists of 800 lines of definitions, theorems and
exercises proposed in the book. It is publicly available\footnote{
https://github.com/authchir/log792-type-systems-formalization} and can be executed with Isabelle/HOL
2014.\footnote{http://isabelle.in.tum.de/website-Isabelle2014/} In this report, we focus on the
definitions and how the theorems are expressed. When relevant, we present both the definitions from
the book and our translation, highlighting and motivating the differences. Some proofs are presented
but not explained. For a deeper insight into the proofs, the best methodology is to study the theory
files in Isabelle.

% $ wc formalization.txt
%  1606  9203 62268 formalization.txt
% $ wc formalization.txt
%   806  4681 33955 formalization.txt

% sed -nr '1h;1!H;${;g;s/\(\*.*\*\)//g;p;}' test.thy
