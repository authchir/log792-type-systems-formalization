\chapter{Introduction}

This bachelor thesis deals with the formalisation, in the Isabelle/HOL theorem prover, of part of
the book \emph{Types and Programming Languages}, hereafter abbreviated \emph{TAPL}, by
Benjamen~C.~Pierce. This work concentrate on four of the languages, ranging from simple arithmetic
expressions to fully fledged lambda-calculus, presented in the first two sections, namely "Untyped
Systems" and "Simple Types".

The main motivation to have chosen this subject is the intersection of personal interested and of
opportunities provided by our intership at the chair for logic and verification at TU München.
Having gradually developped an interest for programming languages in the last years, we were eager
to learn more about the foundations of type theory. Pierce's book stand out as a reference
recommended for a deep introduction to the main elements of the field. Also, as part of our
intership, we worked on the implementation of the (Co)datatype module in the Isabelle/HOL theorem
proover. Having experienced the implementator role, we also wanted to learn about the user role and
about the process of formalization. The choise of this subject for this thesis was thus a logic
consequence.

Before to dig into the realm of formalizations, we first introduce the required background (section
\ref{sec:background}) in lambda-calculus, type systems and Isabelle/HOL. The lambda-calculus is a
core calculus that captures the essentials features of programming languages. That is, there exists
a way to encode high level features such as recursion, datatypes, references, etc. Such calculus can
come, as prorgramming languages, in two variant: typed and untyped. A type system is a syntactic
method to prove the absence of certain behaviors such as the misusage of objects of a certain nature
(e.g. natural numbers, booleans, string of characters, etc.) To formalize those we used
Isabelle/HOL, an interactive theorem prover based on higher order logic. It resembles a programming
language in that one can define datatypes and functions. The difference is that it allows to
postulate properties of the formelly defined elements and to provide machine checked poofs that
those properties are theorems.

The formalizations we performed all have a direct corespondance with chapters from \emph{TAPL}
(section \ref{sec:structure-of-formalization}). We provide one Isabelle/HOL theory file per chapter
and introducte them in the same order. The only exception is the nameless representation of terms
that we introduce earlier because our formalization depent on it while, in the book, it is an
independant subject.

The untyped arithmetic expressions language (section \ref{sec:untyped-arith-expr}) served as a
warm-up to experiment with the general structure of formalizations. It consists of boolean
expressions and natural numbers. This simplicity allows to concentrate on the the important
definitions and to accustoms ourself with the notation. Most of our definitions and theorems closely
follows the ones from the book. The main exceptions being that we explicitly expause some hypothesis
that are implicit in the book and our slightly different definition for the mutli-step evaluation
relation.

The formalization (section \ref{sec:nameless-rep-of-terms}) of the nameless representation of terms,
also known as "de Bruijn indices", was not initially planned but arose from the need to use a
concreate representation for variables in the lambda-calculus. Our formalization closely follows the
book.

While the previous formalizations were either a warm-up or a representation necessity, the untyped
lambda-calculus (section \ref{sec:untyped-lambda-calculus}) is the first core calculus we
formalized. This formalization is the first where we differ from the book in a non-negligible way.
As presented, the evaluation relation assumes that names clashes in variables are automatically
solved by renaming them and, thus, ignore this possibility from there on. Such an assumption is not
accepted for a computer-verified proof. We choosed to use "de Bruijn indices" as representation for
variable to encode this assumption. Also, since the chapter is more focused on explaining the lambda
calculus, it does not contain theorems as the other chapters. Nevertheless, we decide to reprove, or
disprove, the theorems introduce with the arithmetic expressions language.

The typed arithmetic expressions language (section \ref{sec:typed-arith-expr}) is again a warm-up;
this time, to experiment with the formalization of a type system. Our formalization closely follows
the book.

The simply typed lambda-calculus (section \ref{sec:simply-typed-lambda-calculus}) is the second core
calculus that we formalize. Here, we differ significantly from the book. Mainly because of our use
of "de Bruijn indices" but also because of our reprensentation of the typing context, we needed to
adapted some lemmas and replace others. This was quite challenging since we then could not follow
the described proofs anymore and had to find the right assumptions for our lemmas. We belive that
our formalization still respect the spirit of the book since only the helper lemmas changed and the
theorems remains the same.
