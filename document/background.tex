\section{Background}
\label{sec:background}

\subsection{Lambda-Calculus}
\label{sec:background-lambda-calculus}

The lambda-calculus is a minimalist language, where everything is a functions, that can be use as a
core calculus capturing the essential features of complex programming languages. It was formulated
by Alonzo Church \cite{???} in the 1930s as a model of computability. At the same time, Alan Turing
was formulating what is now known as a Turing machine \cite{???} for the same purpose. It was later
proved that both systems are equivalent in expressive power \cite{???}.

As a programming language, it can be intriguing at first because everything reduces to function
abstraction and application. The syntax comprises three sorts of terms: variables, function
abstractions over a variable and applications of a term to an other. Those three constructs are
summarized in the following grammar using a BNF-like notation:
\begin{align*}
  \lambda\text{-term} ::= & \\
    & x && \text{variable} \\
    & \lambda x. t && \text{abstraction} \\
    & t_1 \text{ } t_2 && \text{application}
\end{align*}

Following are a few standard $\lambda$-term shown as examples of how the grammar is actually
used:\footnote{To reduce the need for parenthesis, we use the following standard conventions:
\begin{enumerate*}[label=(\arabic*)]
  \item the body of a $\lambda$-abstraction expands as much as possible to the right and
  \item function application is left associative.
\end{enumerate*}}
\begin{align*}
  & \lambda x. x && \text{identity} \\
  & \lambda x. \lambda y. x && \text{constant} \\
  & \lambda x. \lambda y. x \text{ } y \text{ } y && \text{double application} \\
  & \lambda x. \lambda y. \lambda z. x (y \text{ } z) && \text{function composition}
\end{align*}

Having no built-in constant or primitive operators (e.g. numbers, arithmetic operations,
conditionals, loops, etc.), the only way to "compute" something is by function application, also
known as $\beta$-reduction and noted $\to_\beta$). It consists of replacing every instance of the
abstracted variable in the abstraction body by the effective argument. Following is an example of a
$\lambda$-term that is $\beta$-reduced twice:
\begin{align*}
  (\lambda x. (\lambda y. y) x) ((\lambda w. w) z)
    & \to_\beta (\lambda x. (\lambda y. y) x) z \\
    & \to_\beta (\lambda y. y) z \\
    & \to_\beta z
\end{align*}

This apparently simple operation hide a complex corner case: name clashes. Nothing prevent two
functions from using the same name for their abstracted variable. One can not simply replace every
variable with the same name when performing a substitution but must also take variable scopes into
account. Here is a simple example that demonstrate that a naïve approche can fail to preserve the
semantic of the original $\lambda$-term:

\begin{displaymath}
  (\lambda x. \lambda y. x) y \not\to_\beta \lambda y. y \\
\end{displaymath}

One solution to this problem is to rename function arguments, also known as $\alpha$-equivalence and
noted $=_\alpha$, prior to $\beta$-reduction. Here is a correct $\beta$-reduction for the previous
example:
\begin{align*}
  (\lambda x. \lambda y. x) y
    & =_\alpha (\lambda x. \lambda w. x) y \\
    & \to_\beta \lambda w. y \\
\end{align*}

The difference is important. Under the naïve approche, the $\lambda$-term was wrongly reduced to the
identity function while the correct reduction lead to a function returning the constant $y$.

Many constructions from high level programming languages can be encoded using only those basic
features.  Unary functions are already supported and n-ary functions can be straitforwardly emulate
by having a function return an other function, as was done in the previous examples:

\begin{displaymath}
  \text{n-ary function} \equiv \lambda x_1. \lambda x_2 \dots \lambda x_n. t
\end{displaymath}

An other common construction is a \emph{let binding} which serves to attach an identifier to a
complex expression. It can be emulated in lambda-calculus with a single function abstraction:

\begin{displaymath}
  \text{let } x = y \text{ in } t \equiv (\lambda x. t) y
\end{displaymath}

Although significantly less obvious, it is also possible to express booleans only with functions.
The encoding is based on the idea that any use of booleans can be express with only three
primitives: a constant representing a true value, a constant representing false values and an
operation to choose between two alternatives:
\begin{align*}
  \text{true}
    & \equiv \lambda t. \lambda f. t \\
  \text{false}
    & \equiv \lambda t. \lambda f. f \\
  \text{if } b \text{ then } t \text{ else } e
    & \equiv \lambda b. \lambda t. \lambda f. b \text{ } t \text{ } f
\end{align*}

Using the rules already discussed, it is trivial to show that the following reduction is valid:

\begin{displaymath}
  \text{if true then } x \text{ else } y \to_\beta \dots \to_\beta x
\end{displaymath}

Other encodings exist for constructions such as numbers, list, datatypes, arbitrary recursion, etc.
For a more comprehensive introduction to the subject, \emph{An introduction to Lambda Calculi for
Computer Scientists} \ref{???} can serve as a good starting point.

\subsection{Type Systems}

Type systems are a syntactic method to prove the absence of certain behaviors. They differ from
testing in that they are exhaustive, automatic and compositional. In this context, exhaustiveness
means that each checked invarient is proved to hold for the complete program instead of focusing on
a single unit of code. Automaticness means that the process is non-interactive, i.e. the algorithm
does not need other informations than what it can find in the source code of the program.
Compositional means that proofs for individual components ca be use to satisfy an obligation about
the interaction of the components.

The kind of errors detected depends on the specific type system considered: it can range from
fairly simple to very complex. Examples of simple errors include typographical mistakes, usage of
values of the wrong kind and usage of undefined operations:

\begin{center}
  \begin{tabular}{m{3.5cm} | m{5.5cm}}
    $\text{add} : \mathbb{N} \to \mathbb{N} \to \mathbb{N}$ \newline
    $\text{true} : \mathbb{B}$ \newline
    add true true
    & Error in function application, "add" expects a $\mathbb{N}$ but a $\mathbb{B}$ was provided.
  \end{tabular}
\end{center}

With sufficiently powerfull type systems, specific requirements can be provide as type annotations.
Examples of such include that the output of a sorting function is a permutation of its input, that
the argument of an indexing operation on a list is in a valid range and that two multiplied matrixes
have compatible dimenssions. The more informations one put in the types, the more invalid programs
the type checker can catch, but the more work is required by the programmer to convince the
algorithm that the program fullfill it's specifications. An important design decision when defining
a type system is to find a tradeoff between those two conflicting properties.

Since they are often bundle with the compiler of a programming language and, thus, part of the
normal programming cycle, type systems allow early detection of programming errors. Moreover, the
diagnoses of type checkers can often pointed accuratly the source of the error, at the difference of
run-time tests where the effect of an error can sometime be visible only much farther in the code
when something start to go wrong.

An other important way in which type systems can be use is as an abstraction tool. It is widely
accepted that, in the context of large scale software developpement, programs should be split into
distinct modules that communicate through interfaces. Types are an natural fit to serve as such an
interface. Enven in smaller scale programming, it is usefull to caracterise a datatype not by the
way it is implemented but by the different operations that can be perform on it. This focus on
operations led to the concept of abstract datatypes.

Types can also be use an invaluable maintenance tool. More informally, they serve as a checked
documentation of programs and, being simpler than the computation they caracterise, they can help
to reason about such computation on a higher level. But they can also serve a very practical
purpose, by checking which part of a program is affected by a change. If one decides to change the
arguments of a function or remove a field in a structure, a simple type checking pass will provide
an exhaustive list of the places that need to be updated to cope with the modification.

% Language safety is a very desirable property of programming language that can be achived by type
% systems.
%   * e.g. memory safe
%   * A safe language is completly defined by its programmer manual.
%     * C contains a lot of implementation defined and undefined behaviours.

Due to their static nature, type systems are conservatives in that they will always reject bad
programs at the expence of sometime rejecting good ones. A simple example of such limitation is the
following program that fails to typecheck, even if it happens that the complex boolean expression
will always evaluate to true at runtime:

\begin{displaymath}
  \text{if} <\text{complex-expression}> \text{then } 42 \text{ else } \text{true}
\end{displaymath}

Having only access to static information, a type checker only see that a boolean can take two
different values and, thus, must insure that the program is valide in both cases. In this example,
this implies that both branch of the \texttt{if} must be well-typed and that their type must be
compatible. It is the main goal of type theory scientists to develop type systems in which more more
valid programs are accepted while more invalid programs are rejected.

\subsection{Isabelle/HOL}

\input{Examples}
