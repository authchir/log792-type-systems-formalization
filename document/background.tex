\section{Background}
\label{sec:background}

\subsection{Lambda-Calculus}
\label{sec:background-lambda-calculus}

The $\lambda$-calculus is a minimalistic language, where every value is a function, that can be used
as a core calculus capturing the essential features of complex functional programming languages. It
was formulated by Alonzo Church \cite{church-1936-unsolvable-problem} in the 1930s as a model of
computability. At about the same time, Alan Turing was formulating what is now known as a Turing
machine \cite{turing-1936-on-computable-numbers} for the same purpose. It was later proved that both
systems are equally expressive \cite{turing-1937-computability}.

As a programming language, the $\lambda$-calculus can be intriguing at first because everything
reduces to function abstraction and application. The syntax comprises three sorts of terms:
variables, function abstractions over a variable and applications of a term to an other. Those three
constructs are summarized in the following grammar:
\begin{align*}
  t ::= & \\
    & x && \text{variable} \\
    & \lambda x. \ t && \text{abstraction} \\
    & t_1 \text{ } t_2 && \text{application}
\end{align*}

Below are a few standard $\lambda$-terms shown as examples of how the grammar is actually
used:\footnote{To reduce the need for parentheses, we use the following standard conventions:
\begin{enumerate*}[label=(\arabic*)]
  \item the body of a $\lambda$-abstraction expands as far as possible to the right and
  \item function application is left-associative.
\end{enumerate*}}
\begin{align*}
  & \lambda x. \ x && \text{identity} \\
  & \lambda x. \ \lambda y. \ x && \text{constant} \\
  & \lambda f. \ \lambda x. \ f \ x \ x && \text{double application} \\
  & \lambda f. \ \lambda g. \ \lambda x. \ f \ (g \ x) && \text{function composition}
\end{align*}

The $\lambda$-calculus having no built-in constant or primitive operators (e.g. numbers, arithmetic
operations, conditionals, loops, etc.), the only way to compute a value is by function application,
also known as $\beta$-reduction and denoted by $\to_\beta$). It consists of replacing every instance
of the abstracted variable in the abstraction body by the provided argument. Following is an
example of a $\lambda$-term that is $\beta$-reduced three times:
\begin{align*}
  (\lambda x. \ (\lambda y. \ y) \ x) \ ((\lambda w. \ w) \ z)
    & \: \to_\beta \: (\lambda x. \ (\lambda y. \ y) \ x) \ z \\
    & \: \to_\beta \: (\lambda y. \ y) \ z \\
    & \: \to_\beta \: z
\end{align*}

This apparently simple operation hides a subtle corner case: name clashes. Nothing prevents two
functions from using the same name for their abstracted variable. One cannot simply replace every
variable with the same name when performing a substitution but must also take variable scopes into
account. Here is a simple example that demonstrates that a naive approach can fail to preserve the
semantics of the original $\lambda$-term:
\begin{displaymath}
  (\lambda x. \ \lambda y. \ x) \ y \: \not\to_\beta \: \lambda y. \ y
\end{displaymath}

One solution to this problem is to rename function arguments, also known as $\alpha$-equivalence and
denoted by $=_\alpha$, prior to $\beta$-reduction. Here is a correct $\beta$-reduction for the
previous example:
\begin{align*}
  (\lambda x. \ \lambda y. \ x) \ y
    & \: =_\alpha \: (\lambda x. \ \lambda w. \ x) \ y \\
    & \: \to_\beta \: \lambda w. \ y
\end{align*}

The difference is important: under the naive approach, the $\lambda$-term was wrongly reduced to the
identity function while the correct reduction lead to a function returning the constant $y$.

Many constructions from high level programming languages can be encoded using only those basic
features. Unary functions are already supported and $n$-ary functions can be straightforwardly
emulated by having a function return another function, as was done in the previous examples:
\begin{displaymath}
  n\text{-ary function} \: \equiv \: \lambda x_1. \ \lambda x_2. \ \cdots \ \lambda x_n. \ t
\end{displaymath}

Another common construction is a \emph{let binding} which serves to attach an identifier to a
complex expression. It can be emulated in the $\lambda$-calculus with a single function abstraction:
\begin{displaymath}
  \text{let } x = y \text{ in } t \: \equiv \: (\lambda x. \ t) \ y
\end{displaymath}

Although significantly less obvious, it is also possible to express Booleans only with functions.
The encoding is based on the idea that any use of Booleans can be expressed with only three
primitives: a constant representing a true value, a constant representing a false value and an
operation to choose between two options:
\begin{align*}
  \text{true}
    & \: \equiv \: \lambda t. \ \lambda f. \ t \\
  \text{false}
    & \: \equiv \: \lambda t. \ \lambda f. \ f \\
  \text{if } b \text{ then } t \text{ else } e
    & \: \equiv \: \lambda b. \ \lambda t. \ \lambda f. \ b \ t \ f
\end{align*}

Using the rules already discussed, it is easy to show that the following reduction is
valid:\footnote{Multiple consecutive $\beta$-reductions are denoted with $\to_\beta^*$.}
\begin{align*}
  \text{if true then } x \text{ else } y
    & \equiv \: (\lambda b. \ \lambda t. \ \lambda f. \ b \ t \ f) \ true \ x \ y \\
    & \to_\beta^* \: true \ x \ y \\
    & \equiv \: (\lambda t. \ \lambda f. \ t) \ x \ y \\
    & \to_\beta^* \: x
\end{align*}

Other encodings exist for constructions such as numbers, list, datatypes, arbitrary recursion, etc.
For a more comprehensive introduction to the subject, Hankin's monograph \cite{hankin-2004-ILCCS} is
a good starting point.

\subsection{Type Systems}

Type systems are a syntactic method to prove the absence of certain erroneous behaviors. They differ
from testing in that they are exhaustive and compositional. In this context, exhaustiveness means
that each checked invariant is proved for the complete program instead of focusing on a single unit
of code. Compositionality means that proofs for individual components can be use to discharge an
obligation about the interaction of the components.

The kinds of errors detected depends on the specific type system considered: they can range from
fairly simple to very complex. Examples of simple errors include typographical mistakes, usage of
values of the wrong kind and usage of undefined operations:\footnote{The syntax $T_1 \to T_2$ is the
standard notation for the type of a function with domain $T_1$ and codomain $T_2$. The $\to$
operator being right-associative, the type $T_1 \to (T_2 \to T_3)$ can be written without
parentheses: $T_1 \to T_2 \to T_3$.}

\begin{center}
  \begin{tabular}{m{3.5cm} | m{5.5cm}}
    $\text{add} : \mathbb{N} \to \mathbb{N} \to \mathbb{N}$ \newline
    $\text{true} : \mathbb{B}$ \newline
    add true true
    & Error in function application, ``add'' expects a $\mathbb{N}$ as first argument but a
    $\mathbb{B}$ was provided.
  \end{tabular}
\end{center}

With sufficiently powerful type systems, specific requirements can be provided as type annotations.
Examples of such include that the output of a sorting function is a permutation of its input, that
the argument of an indexing operation on a list is in a valid range and that two multiplied matrices
have compatible dimensions. The more information one puts in the types, the more invalid programs
the type checker can catch. The interactive theorem prover Coq is based on this idea: each
transformation, including theorems, is internally a type-transformation. The drawback of such
expressive type systems is that more work is required by the programmer to convince the type
checking algorithm that the program fulfills its specification.  An important design decision when
defining a type system is to find a tradeoff between those conflicting goals.

Since they are often bundled with the compiler of a programming language and, thus, part of the
normal programming cycle, type systems allow early detection of programming errors. Moreover, the
diagnoses of type checkers can often pointed accurately the source of the error, unlike run-time
tests where the effect of an error can sometime be visible only much further in the code when
something starts to go wrong.

Another important way in which type systems can be used is as an abstraction tool. Large scale
software generally consist of modules that communicate through interfaces. Types are a natural fit
to serve as such an interface. Even in smaller scale programming, it is useful to characterize a
datatype not by the way it is implemented but by the different operations that can be perform on it.
This focus on operations led to the concept of abstract datatypes.

Types can also be used an invaluable maintenance tool. They serve as a checked documentation of
programs and, being simpler than the computations they characterize, they can help to reason about
such computations on a higher level. But they can also serve a very practical purpose, by checking
which part of a program is affected by a change. If one decides to change the arguments of a
function or to remove a field in a structure, a simple type checking pass will provide an exhaustive
list of the places that must be updated.

Due to their static nature, type systems are normally conservative in that they will always reject
bad programs at the expense of sometime rejecting good ones. A simple example of such limitation is
the following program that fails to type-check, even though the Boolean expression will always
evaluate to true at runtime:
\begin{displaymath}
  \begin{tabular}{m{4.5cm} | m{6cm}}
    if true then 42 else true
    & Type mismatch in conditional expression, the type of the ``then'' branch is $\mathbb{N}$ while
    the type of the ``else'' branch is $\mathbb{B}$.
  \end{tabular}
\end{displaymath}

Having only access to static information, a type checker only see that a Boolean can take two
different values and, thus, must ensure that the program is valid in both cases. In this example,
this implies that both branches of the ``if'' must be well typed and that their types must be
compatible. It is the main goal of researchers on type systems to develop systems in which more
valid programs are accepted while more invalid programs are rejected.

\subsection{Isabelle/HOL}

\input{Examples}
