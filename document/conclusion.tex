\section{Conclusion}

We formalized a number of languages presented in the \emph{Types and Programming Languages} with the
Isabelle/HOL interactive theorem prover. We started with the a simple arithmetic language of
booleans and natural numbers. We continued with the nameless representation of terms for the
lambda-calculus, which we used as a basis for the pure untyped lambda calculus. For those languages,
we proved the determinacy of evaluation, the relation between values and normal form, the uniqueness
of normal form and the termination or non-termination of evaluation.

We then revisited the both languages and augment them with type systems. We proved the uniqueness of
types and the safety of the languages through the progress and preservation theorems. We also
demonstrated that the addition of types did not changed the semantic of the lambda-calculus by
proving that types can be erase without affecting the evaluation of terms.

The formalization process can be separate in three main activies: defining the primitives and the
formulating the theorems and finally the proving per se. In retrospective, the first and second
activities are the more important an difficult. Once the right abstractions and the correct
formulation for a theorem have been found, the proofs were usually fairly simple. On the contrary, a
wrong abstraction or hypothesis have lead use to difficult, or even impossible, to prove properties.

In this report, we consentrated our attention on the definitions on how theorems were expressed,
highlighting the differences with the book. The complete Isabelle/HOL theories provided along this
report contain more examples, exercices and less important theorems.
