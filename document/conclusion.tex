\section{Conclusion}

In this thesis, we formalized a number of languages presented in \emph{Types and Programming
Languages} using the Isabelle/HOL interactive theorem prover. We started with a simple arithmetic
language of Booleans and natural numbers. We continued with the nameless representation of terms for
the $\lambda$-calculus, which we used as a basis for the pure untyped $\lambda$-calculus. For those
languages, we proved the determinacy of evaluation, the relation between values and normal form, the
uniqueness of normal form and the termination, or non-termination, of evaluation.

We then revisited both languages and augmented them with type systems. We proved the uniqueness of
types and the safety of the languages through the progress and preservation theorems. We also
demonstrated that the addition of types did not changed the semantics of the $\lambda$-calculus by
proving that types can be erased without affecting the evaluation of terms.

A formalization can be separated in three main elements: definitions, properties involving these
definitions and computer-checked proofs that these properties hold. In retrospective, expressing the
definitions and properties was the most important and difficult activity. Once the right
abstractions and the correct formulations for theorems were found, the proofs were usually fairly
simple: a good definition is worth three theorems! Conversely, a wrong abstraction or hypothesis
have led us to theorems very difficult to prove, or even to properties that ended up not being
theorems at all.

In this report, we focused our attention on the definitions and theorems, highlighting the
differences with the book. The complete Isabelle/HOL theories provided along with this
report\footnote{\url{https://github.com/authchir/log792-type-systems-formalization}} contain more
examples, exercises and less important theorems.
