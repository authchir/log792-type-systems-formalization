\section{Structure of the Formalization}
\label{sec:structure-of-formalization}

In this thesis, we formalize six chapters of the first two sections of \emph{TAPL}. Figure
\ref{fig:TAPL-toc} presents the table of contents of those two sections --- the formalized chapters
are in bold --- and Figure \ref{fig:TAPL-dependencies} presents the dependencies between the
chapters; a normal arrow imply a direct dependency while a dashed arrow only imply that the
knowledge learned in one chapter is reused.

\begin{figure}[h]
  \footnotesize
  \centering
  \begin{varwidth}{\textwidth}
    \begin{enumerate}[label=\Roman*]
      \itemsep 1pt
      \item Untyped Systems \hfill
        \begin{enumerate}[label=§ \arabic*]
          \setcounter{enumii}{2}
          \item \textbf{Untyped Arithmetic Expressions}
          \item An ML Implementation of Arithmetic Expressions
          \item \textbf{The Untyped Lambda-Calculus}
          \item \textbf{Nameless Representation of Terms}
          \item An ML Implementation of the Lambda-Calculus
        \end{enumerate}
      \item Simple Types \hfill
        \begin{enumerate}[label=§ \arabic*]
          \setcounter{enumii}{7}
          \item \textbf{Typed Arithmetic Expressions}
          \item \textbf{Simply Typed Lambda-Calculus}
          \item An ML implementation of Simple Types
          \item Simple Extensions
          \item Normalization
          \item References
          \item Exceptions
        \end{enumerate}
    \end{enumerate}
  \end{varwidth}
  \caption{Part I and II of \emph{TAPL}}
  \label{fig:TAPL-toc}
\end{figure}

\begin{figure}[h]
  \begin{center}
    \begin{tikzpicture}[>=triangle 45]
      \draw (0,3) rectangle (1,4) node[pos=0.5]{6};
      \draw (1.5,1.5) rectangle (2.5,2.5) node[pos=0.5]{5};
      \draw (3,0) rectangle (4,1) node[pos=0.5]{3};
      \draw (3,3) rectangle (4,4) node[pos=0.5]{9};
      \draw (4.5,1.5) rectangle (5.5,2.5) node[pos=0.5]{8};
      \draw [->] (1,3) -- (1.5,2.5); % 6 -> 5
      \draw [->, dashed] (2.5,1.5) -- (3,1); % 5 -> 3
      \draw [->] (3,3) -- (2.5,2.5); % 9 -> 5
      \draw [->, dashed] (4,3) -- (4.5,2.5); % 9 -> 8
      \draw [->] (4.5,1.5) -- (4,1); % 8 -> 3
    \end{tikzpicture}
  \end{center}
  \caption{Dependencies between the chapters of TAPL}
  \label{fig:TAPL-dependencies}
\end{figure}

The formalization closely follows the structure of the book. We provide one Isabelle theory file per
chapter and mainly introduce them in the same order. The only exception is the chapter on "Nameless
Representation of Terms". In the book, it is treated as a completely separate issue from the
chapters on $\lambda$-calculus (i.e. an encoding that can be useful for an implementation). They
present it as a concrete alternative to the implicit $\alpha$-conversion they assume in their
proofs, which is nice for humans but not rigorous enough for a computer. Even though a formalization
is different from an implementation it have some similar requirements. Since we chose this nameless
representation for our formalization, we must diverge from the book and introduce this subject
before the untyped $\lambda$-calculus. Figure \ref{fig:thys-dependencies} presents the dependencies
between our Isabelle theory files.

\begin{figure}[h]
  \begin{center}
    \begin{tikzpicture}[>=triangle 45]
      \draw (0,3) rectangle (1,4) node[pos=0.5]{6};
      \draw (1.5,1.5) rectangle (2.5,2.5) node[pos=0.5]{5};
      \draw (3,0) rectangle (4,1) node[pos=0.5]{3};
      \draw (3,3) rectangle (4,4) node[pos=0.5]{9};
      \draw (4.5,1.5) rectangle (5.5,2.5) node[pos=0.5]{8};
      \draw [->, dotted] (1.5,2.5) -- (1,3); % 5 -> 6
      \draw [->, dashed] (2.5,1.5) -- (3,1); % 5 -> 3
      \draw [->] (3,3) -- (2.5,2.5); % 9 -> 5
      \draw [->, dashed] (4,3) -- (4.5,2.5); % 9 -> 8
      \draw [->, dotted] (4.5,1.5) -- (4,1); % 8 -> 3
    \end{tikzpicture}
  \end{center}
  \caption{Dependencies between the theory files}
  \label{fig:thys-dependencies}
\end{figure}

It was possible to directly base the typed arithmetic expressions on untyped arithmetic expression
by importing the theory and reusing its definition. This is represented as a dotted arrow in figure
\ref{fig:thys-dependencies}. This reuse is possible because, for this language, types are external
to the representation of terms. By contrast, for the typed $\lambda$-calculus, we need to alter the
representation of terms to add the typing annotation on abstraction variables, thus preventing the
reuse of the untyped $\lambda$-calculus theory.
