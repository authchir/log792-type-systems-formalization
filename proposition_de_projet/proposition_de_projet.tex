\documentclass[a4paper, oneside, 12pt, titlepage, draft]{article}

\usepackage[utf8]{inputenc}
\usepackage[T1]{fontenc}
\usepackage[canadien]{babel}
\usepackage{glossaries}
\usepackage{float}
\usepackage[inline]{enumitem}

\makeindex
\makeglossaries
\newacronym{HOL}{HOL}{Higher Order Logic}

\floatstyle{boxed}
\restylefloat{figure}

\begin{document}

\title{Formalisation de systèmes de types à l'aide d'Isabelle/HOL}
\author{Martin Desharnais}
\maketitle

\tableofcontents
%\listoftables
%\listoffigures
%\printglossaries

\cleardoublepage

\section{Problématique et contexte}

% \emph{Quel sont les problèmes généraux et spécifiques que vous cherchez à solutionner – totalement
% ou en partie? Quel est le contexte de ces problèmes? Quelle est la situation actuelle?}

Ce projet s'intéresse à l'étude des systèmes de types, dans le contexte des langages de
programmation, dont voici une définition :

\begin{quotation} % Benjamin C. Pierce, Types and Programming Languages, §1.1
  Un système de types est une méthode syntaxique tractable pour prouver l'absence de certains
  comportement des programmes par la classification des phrases selon le genre de valeurs qu'elles
  calculent.
\end{quotation}

L'objectif est donc de garantir, sans l'exécuter, qu'un programme est exempt de certaines erreurs
telles qu'une faute typographique dans un nom de variable, l'appel d'une fonction non supporté dans
un certain context ou encore une tentative de diviser un nombre par une chaine de caractères. De
tels exemples peuvent sembler simplistes mais sont très fréquentes et peuvent avoir des conséquences
désastreuses : une incohérence informatique entre les systèmes d'unités métrique et anglo-saxon a
provoqué la destruction du Mars Climate Orbiter en 1999. Bien sur, tous les défauts ne peuvent pas
être décelés par un système de types. Cependant, il en existe un très grand nombre, de divers
niveaux d'expressivité et de complexité, qui permettent de détecter un éventail varié d'erreurs.

Lors du développement d'un nouveau système de type, un ensemble de preuves est réalisé afin de
démontrer que celui-ci respecte ses objectifs. L'étude de ces systèmes ainsi que des
preuves qui les accompagnent est le sujet du présent projet.

\section{Objectifs du projet}

% \emph{Dans cette partie, vous décrivez précisément  ce que vous allez accomplir dans le projet.
% Quels sont les résultats attendus de votre projet? Quelles sont les retombées du projet?
% Assurez-vous que vos objectifs sont "testables", c'est-à-dire que vous serez capable de démontrer
% de façon convaincante, à la fin du projet, que les objectifs ont été atteints. Assurez-vous aussi
% que vos objectifs sont en lien avec la problématique énoncée plus haut.}

% Learning (Paper-Proof of Type Systems properties & Isabelle/HOL)
% Validate Paper-Proofs
% Clarification (Paper-Proof & Isabelle/HOL)

Les objectifs de ce projet sont quadruples :
\begin{enumerate*}[label=\arabic*)]
  \item s'initier à la formalisation avec l'assistant de preuve Isabelle/HOL;
  \item s'initier à la théorie des types;
  \item valider les preuves manuelles existantes et
  \item clarifier les cas limites de ces dernières.
\end{enumerate*}

\subsection{S'initier à la formalisation Isabelle/HOL}

Deux grandes catégories d'outils sont disponnibles pour effectuer une formalisation : les prouveurs
automatiques et les assistants de preuve interactifs. Quelque soit l'outil utilisé, il faut définir
un context de travail et une équation que l'on veut prouver. La différence est que, dans le premier
cas, l'outil tentera de trouver une preuve entièrement automatiquement alors que, dans le second
cas, il faut travailler interactivement avec l'outils pour prouver le théorème.

Isabelle/HOL est un assistant de preuve interactif utilisant la logique d'ordre supérieure. Il pemet
de spécifier des formules mathématiques, algorithmes et objets dans un langage déclaratif,
fonctionel et typé. Il est alors possible de spécifier des propriété sur l'interaction entre les
divers éléments. Une fois le système formalisé, il est possible d'en extraire du code exécutable
correspondant aux spécifications. Ce projet sera l'occasion de s'initier à la définition d'un
système formel et aux preuves interactives à l'aide d'Isabelle/HOL.

\subsection{S'initier à la théorie des types}

Plusieurs des langages de programmations dominant actuellement ont un système de type appliqué à la
compilation. La majorité des programmeurs sont donc familiés avec le concept. Malheureusement, les
systèmes de types présent dans ces langages sont généralement simples, imposent nombre de
contraintes à leur utilisateurs et n'offrent qu'un nombre limité de garanties en retour. Certaines
de ces limitations ont été mitigées dans de nouvelles versions du langage (e.g. l'ajout des types et
fonctions génériques en Java).

D'un autre coté, des alternatives plus expressives et plus puissantes sont bien connues ou bien
actuellement en développement par les acteurs du milieu. Ce projet sera l'occastion de consolider
les acquis fondamentaux et d'en apprendre plus sur les concepts plus avancés de la théorie des types

\subsection{Valider les preuves manuelles existantes}

La théorie des types étant un sujet de recherche très actif depuis plusieurs dizaines d'années, un
grand nombre de publications décrivent les caractéristiques de différents systèmes de type.
Cependant, une preuve manuelle étant validée par des être humains, il est toujours possible que des
erreurs s'y soient glissées. La formalisation de celles-ci à l'aide d'un assistant de preuve permet
de valider, sous réserve que l'assistant de preuve est correct, qu'aucune erreur logique n'est
présente. S'il s'avérait qu'une erreur soit découverte dans le cadre de ce projet, l'information
serait transmise à l'auteur initial afin de l'informer de la situation.

\subsection{Clarifier les cas limites des preuves manuelles}

Les propriétés énoncées et prouvées manuellement semblent souvent évidentes dès lors qu'elles sont
appliquées à un exemple concret. Cette méthode de visualisation a toutefois ses limites puisque
certaines constructions plus complexes peuvent entrainer des résultats inattendus. La formalisation
de ces propriétés à l'aide d'un assistant de preuve oblige son auteur à considérer la liste
exhaustive des constructions du langage et permet ainsi d'acquérir un meilleur compréhension de la
propriété et des cas limites. Cette technique a été utilisé avec succès afin d'enseigner
l'ingénierie logiciels à des étudiants de premier cycle. \emph{CITATIONS NEED}

\section{Méthodologie}

% \emph{Expliquez comment vous allez atteindre les objectifs du projet. Ce sont les activités
% (analyse, conception, mesure, tests, gestion de la configuration, revues, etc.) et les
% responsabilités identifiées dans cette section qui guideront l’affectation des responsabilités aux
% membres de l’équipe et aussi à l’établissement de votre échéancier de travail.}
%
% \emph{Note : une approche itérative est recommandée.}

L'ouvrage de référence de ce projet est le livre « Types and Programming Languages » de Benjamin C.
Pierce. Ce livre est composé de six sections : les systèmes non-typés, les types simples, le
sous-typage, les types récursifs, le polymorphisme et les systèmes d'ordre supérieur. Chaque section
est composé de plusieurs chapitres décrivant un système de type bonifiant un système décrit
précédemment en lui adjoignant un concept supplémentaire. Les figures \ref{fig:TAPL-section-1} et
\ref{fig:TAPL-section-2} présentent les chapitres des deux premières sections sur lesquelles se
concentrera ce projet --- ceux en gras sont ceux qui seront formalisés--- et la figure
\ref{fig:TAPL-chapter-dependencies} présente les dépendances entre ceux-ci.

Ces chapitres ont été choisie car ils décrivent des langages et leurs théorèmes au lieu d'en
expliquer la théorie ou d'en présenter une implémentation. De plus, ils culminent au lambda-calcul
symplement typé qui a la propriété de pouvoir représenter \footnote{Ceci n'est pas tout à fait
exacte puisqu'il faudrait y ajouter la fonctionnalité de pouvoir communiquer avec l'environement
d'exécution afin de pouvoir faire des opérations comme lire ou écrire dans un fichier, effectuer de
la communication interprocessus, etc.} la majorité des langages de programmations existants.

\begin{figure}[h]
  \begin{enumerate}[label=§ \arabic*]
      \setcounter{enumi}{2}
    \item \textbf{Expressions arithmétiques non-typées}
    \item Une implémentation en ML des expressions arithmétiques
    \item \textbf{Le lambda-calcul non-typé}
    \item Représentation non-nommé des termes
    \item Une implémentation en ML du lambda-calcul
  \end{enumerate}
  \caption{Section I du livre de référence --- Les systèmes non-typés}
  \label{fig:TAPL-section-1}
\end{figure}

\begin{figure}[h]
  \begin{enumerate}[label=§ \arabic*]
      \setcounter{enumi}{7}
    \item \textbf{Expressions arithmétiques typées}
    \item \textbf{Le lambda-calcul simplement typé}
    \item Une implémentation en ML des types simples
    \item Extensions simples
    \item Normalisation
    \item Références
    \item Exceptions
  \end{enumerate}
  \caption{Section II du livre de référence --- Les types simples}
  \label{fig:TAPL-section-2}
\end{figure}

\begin{figure}[h]
  \begin{center}
    \setlength{\unitlength}{1cm}
    \begin{picture}(4,4.5)
      \thicklines
      \put(0,0){\framebox(1,1){3}}
      \put(0,3){\framebox(1,1){8}}
      \put(0.5,3){\vector(0,-1){2}}
      \put(3,0){\framebox(1,1){5}}
      \put(3,0.5){\vector(-1,0){2}}
      \put(3,3){\framebox(1,1){9}}
      \put(3,3.5){\vector(-1,0){2}}
      \put(3.5,3){\vector(0,-1){2}}
    \end{picture}
  \end{center}
  \caption{Dépendances entre les chapitres}
  \label{fig:TAPL-chapter-dependencies}
\end{figure}

Le projet formalisera séquentiellement les différents chapitres en se basant sur le travail
des chapitres précédents. Pour cette raison, le premier chapitre sera le plus long à formaliser :
tout devant être fait à partir de zéro. Pour les chapitres suivants, la première étape sera
d'importer la formalisation des dépendances et de la modifier pour inclure les nouveau concepts
introduits.

Chacune des formalisations se fera en quatre étapes :

\begin{enumerate}
  \item Lecture attentive du chapitre et compréhension générale des concepts énoncés;
  \item Définition dans Isabelle/HOL des structures nécessaires à la formalisation;
  \item Preuve des différents lemmes et théorèmes avancés ansi que, optionellement, les exercices;
  \item Simplification des définitions et preuves.
\end{enumerate}

L'objectif principal du projet étant de formaliser le chapitre 9 et ce genre de formalisation étant
un projet ambitieux, un certain nombre d'actions pouraient être entreprises si le respet des
échéancier s'avère menacé :

\begin{enumerate}
  \label{lower-objectives}
  \item Ne prendre en compte les exercices proposés;
  \item Définir certains lemmes comme des axiomes au lieu de les prouver;
  \item Sauter les chapitres 3 ou 4 afin de passer directement au chapitre 9.
\end{enumerate}

\section{Livrables et planification}

\subsection{Description des artéfacts}

\emph{Cette section identifie les artefacts qui seront produits durant le projet, ainsi que la
planification de leur réalisation.}

\begin{description}
  \item[Fiche de renseignement]
    Formulaire fournissant le titre du projet, un cours résumé ainsi que les noms des étudiants
    impliqués et du professeur superviseur.
  \item[Proposition de projet]
    Document présentant la problématique du projet, les objectifs, la méthodologie, les livrables,
    le plan de travail, les risques ainsi que les techniques et outils utilisés.
  \item[Rapport d'étape]
    Document présentant une version étoffée et mise à jour de la proposition de projet, ainsi qu'une
    version partielle du rapport technique.
  \item[Diapositives de la présentation orale]
    Diapositives utilisées pour la présentation orale finale du projet.
  \item[Rapport final]
    Document présentant la problématique du projet, les objectifs, la méthodologie employée, les
    hypothèses, les résultats, l'analyse des résultats, les conclusions, les recommandations et les
    références.
  \item[Théories Isabelle/HOL]
    Fichiers sources utilisés par l'assistant de preuve Isabelle/HOL pour sauvegarder les
    définitions et théorèmes formalisés au cours de ce projet.
\end{description}

\subsection{Planification}

% Voir Annexe A. Commentez le tableau de l’annexe A. Une approche itérative est recommandée.

\section{Risques}

% \emph{Vous devez énumérer les difficultés probables que vous pourriez rencontrer et qui pourraient
% avoir un impact sur la réalisation de votre projet. Vous devez expliquer dans un tableau chaque
% risque identifié, son impact ainsi que les moyens que vous allez mettre en œuvre pour le gérer et
% atténuer sa probabilité ou son impact.}
%
% \emph{La forme d'un risque devrait être négative. C'est-à-dire, un risque est un événement que l'on
% veut éviter. Par exemple "expérience du client" n'est pas un risque tandis que "manque
% d’expérience du client" en serait un. Voici un exemple de risque : "Un passager est blessé dans
% une voiture lors d'une collision." Une mitigation à un risque est la stratégie qui vise à éviter
% que l'événement négatif se produise. Par exemple : "Le passager porte une ceinture de sécurité et
% la voiture est équipée de coussin de sécurité gonflable." Soyez spécifique à votre projet. "Manque
% de temps pour finir le projet" et "Portée trop ambitieuse" sont des risques pour n'importe quel
% projet. Il n’est pas mauvais d’avoir des risques génériques, mais il est important d’aussi trouver
% plusieurs risques spécifiques à votre projet, de même que de trouver des mitigations qui sont
% spécifiques à votre projet.}
%
% \emph{Si vous manquez d'inspiration, vous pouvez consulter la liste LOG\_GTI\_792\_Generic Software
% Risk Factors.xls, disponible sur le site Web du cours sous la rubrique ‘Gabarits et guides’.
% Toutefois, faire attention de ne pas ainsi identifier des risques et mitigations qui sont
% applicables à la majorité des projets.}

Un certain nombre de risques sont identifiés dans le tableau \ref{tab:risks}

\begin{table}[!h]
  \small
  \begin{tabular}{|p{3.5cm}|p{3.5cm}|p{1.5cm}|p{3.5cm}|}
    \hline
    \textbf{Risque} &
    \textbf{Impact} &
    \textbf{Proba-bilité} &
    \textbf{Mitigation / \newline atténuation} \\
    \hline
    Manque d'expérience avec la théorie des types &
    Monopolise du temps pour apprendre la théorie. &
    Faible &
    Étudier attentivement l'ouvrage de référence. \\
    \hline
    Manque d'expérience avec l'assistant de preuve Isabelle/HOL &
    Monopolise du temps pour apprendre le fonctionnement de l'outil. &
    Forte &
    Étudier attentivement et se référer au besoin à la la documentation de l'outil. \\
    \hline
    Manque d'expérience en formalisation &
    Monopolise du temp pour apprendre la méthodologie. &
    Forte &
    S'informer auprès des chercheurs expérimentés de la chaire de recherche. \\
    \hline
    Objectifs trop ambitieux &
    Ne pas réaliser toutes les formalisations prévues. &
    Moyenne &
    Un certain nombre de mitigations sont décrites à la page \pageref{lower-objectives} de la
    section \ref{lower-objectives}. \\
    \hline
  \end{tabular}
  \caption{Risques et mitigations}
  \label{tab:risks}
\end{table}

\section{Techniques et outils}

\begin{description}
  \item[Isabelle] Système générique pour l'implémentation de formalismes logiques.
  \item[Isabelle/HOL] Spécialisation d'Isabelle pour la logique d'ordre supérieur (\gls{HOL}).
  \item[Isabelle/Isar] Langage structuré permettant d'écrire des preuves plus lisibles.
  \item[Isabelle/jEdit] Environement de développement intégré pour Isabelle basé sur jEdit.
  \item[Sledgehammer] Outil appliquant des prouveurs automatiques de théorèmes ainsi que des
    solveurs de « satisfaisabilité modulo théories »
\end{description}

\section{Références}

\emph{Vos références initiales. Par exemple: documents de projet existants, compilation de patrons,
documentation sur la technologie, etc.}

\begin{itemize}
  \item Benjamin C. Pierce, Types and Programming Languages
  \item Benjamin C. Pierce, Using a Proof Assistant to Teach Programming Language Foundations
  \item Jasmin Christian Blanchette, Hammering Away : A User's Guide to Sledgehammer for Isabele/HOL
  \item Rex Page, Carl Eastlund, Matthias Felleisen, Functional Programming and Theorem Proving for
    Undergraduates: A Progress Report
  \item Tobias Nipkow, Programming and Proving in Isabelle/HOL
  \item Tobias Nipkow, Teaching Semantics with a Proof Assistant: No more LSD Trip Proofs
  \item Tobias Nipkow, Lawrence C. Paulson, Markus Wenzel, Isabelle/HOL : A Proof Assistant for
    Higher-Order Logic
\end{itemize}

\section{Annexe A : Plan de travail}

\emph{Le tableau suivant présente la planification pour la réalisation des tâches ou artefacts décrits
précédemment.}

\emph{Note : il faut compléter la liste des tâches et des jalons. Assurez-vous que vos tâches sont
exprimées en termes d'activités d'ingénierie, par exemple analyse, architecture, conception,
implémentation, etc, et non en termes de production d'artéfacts (ex: rédaction document X). Les
artéfacts affectés par une activité d'ingénierie sont indiqués dans la colonne
"Livrables/Artéfacts". Si vous utilisez une approche itérative, rendez vos itérations visibles
dans ce plan. Si le projet est réalisé en équipe, ajoutez une colonne d'effort pour chaque membre
de l'équipe.}

\end{document}
