\documentclass[a4paper, oneside, 12pt, titlepage, draft]{article}

\usepackage[utf8]{inputenc}
\usepackage[T1]{fontenc}
\usepackage[canadien]{babel}
\usepackage{glossaries}
\usepackage{float}
\usepackage[inline]{enumitem}

\makeindex
\makeglossaries
\newacronym{HOL}{HOL}{Higher Order Logic}

\floatstyle{boxed}
\restylefloat{figure}

\begin{document}

\title{Formalisation de systèmes de types à l'aide d'Isabelle/HOL}
\author{Martin Desharnais}
\maketitle

\tableofcontents
%\listoftables
%\listoffigures
%\printglossaries

\cleardoublepage

\section{Problématique et contexte}

% Quel sont les problèmes généraux et spécifiques que vous cherchez à solutionner – totalement ou en
% partie? Quel est le contexte de ces problèmes? Quelle est la situation actuelle?

Ce projet s'intéresse à l'étude des systèmes de types dans le contexte des langages de
programmation. Voici une définition possible :

\begin{quotation} % Benjamin C. Pierce, Types and Programming Languages, §1.1
  Un système de type est une méthode syntaxique ??tractable?? pour prouver l'absence de certains
  comportement des programmes par la classification des phrases selon le genre de valeurs qu'elles
  calculent.
\end{quotation}

L'objectif d'un système de type est donc de garantir, sans exécution, qu'un programme est exempt de
certaines erreurs (e.g. l'utilisation de deux systèmes d'unité, sans effectuer les conversions
appropriées, qui a entrainé la destruction du Mars Climate Orbiter). Ainsi, il existe un très grand
nombre de systèmes de types de divers niveau d'expressivité et de complexité. Lors du développement
d'un système de type, un ensemble de preuves est réalisé afin de démontrer que le nouveau système
respecte ses objectifs.

L'étude de ces systèmes ainsi que des preuves qui les accompagnent sont le sujet du présent projet.

\section{Objectif du projet}

% Dans cette partie, vous décrivez précisément  ce que vous allez accomplir dans le projet. Quels
% sont les résultats attendus de votre projet? Quelles sont les retombées du projet? Assurez-vous
% que vos objectifs sont "testables", c'est-à-dire que vous serez capable de démontrer de façon
% convaincante, à la fin du projet, que les objectifs ont été atteints. Assurez-vous aussi que vos
% objectifs sont en lien avec la problématique énoncée plus haut.

% Learning (Paper-Proof of Type Systems properties & Isabelle/HOL)
% Validate Paper-Proofs
% Clarification (Paper-Proof & Isabelle/HOL)

Les objectifs de ce projet sont quadruples :
\begin{enumerate*}[label=\arabic*)]
  \item s'initier à la formalisation avec l'assistant de preuve Isabelle/HOL;
  \item s'initier à la théorie des types;
  \item valider les preuves manuelles existantes et
  \item clarifier les cas limites de ces dernières.
\end{enumerate*}

\subsection{S'initier à la formalisation Isabelle/HOL}

Isabelle/HOL est un assistant de preuve utilisant la logique d'ordre supérieure. Il permet de
spécifier des formules mathématiques, relations et algorithmes, de prouver des propriétés de ces
derniers et de générer du code exécutable correspondant à ces spécification. Ce projet se
concentrera sur la spécification de systèmes de types et l'écriture de preuves des différentes
propriétés qui y sont associées.

\subsection{S'initier à la théorie des types}

La théorie des types est un domaine d'étude à l'intersection de la logique, des mathématiques et de
la philosophie. Son application à l'informatique permet de vérifier, sans exécution, qu'un programe
respecte certaines propriétées. Les langages de programmtions dominants actuellement ne fournissent
qu'un nombre limité de garanties. Cependant, des alternatives plus expressives et plus puissantes
sont connues ou bien en développement.

\subsection{Valider les preuves manuelles existantes}

La théorie des types étant un sujet de recherche très actif depuis plusieur dizaines d'années, un
grand nombre de publications décrivent les caractéristiques de différents systèmes de type.
Cependant, une formalisation manuelle étant valider par des être humains, il est toujours possible
que des erreurs s'y soient glissées. La formalisation de celles-ci à l'aide d'un assistant de preuve
permet de valider, sous réserve de l'assistant de preuve est correct, qu'aucune erreur logique n'est
présente.

\subsection{Clarifier les cas limites des preuves manuelles}

Les propriétés énoncées et prouvées manuellement semblent souvent évidentes dès lors qu'elles sont
appliquées à un exemple concret. Cette méthode de visualisation a toutefois ses limites puisque
certaines constructions plus complexes peuvent entrainer des résultats inattendus. La formalisation
de ces propriétés à l'aide d'un assistant de preuve oblige son auteur à considérer la liste
exhaustive des constructions du langage et permet ainsi d'acquérir un meilleur compréhension de la
propriété et des cas limites.

\section{Méthodologie}

% Expliquez comment vous allez atteindre les objectifs du projet. Ce sont les activités (analyse,
% conception, mesure, tests, gestion de la configuration, revues, etc.) et les responsabilités
% identifiées dans cette section qui guideront l’affectation des responsabilités aux membres de
% l’équipe et aussi à l’établissement de votre échéancier de travail.
%
% Note : une approche itérative est recommandée.

L'ouvrage de référence de ce projet est le livre « Types and Programming Languages » de Benjamin C.
Pierce. Ce livre est composé de six sections : les systèmes non-typés, les types simples, le
sous-typage, les types récursifs, le polymorphisme et les systèmes d'ordre supérieur. Chaque section
est composé de plusieurs chapitres décrivant un système de type bonifiant les systèmes décrits
précédemments en leur adjoignant un concept supplémentaire. Les figures \ref{fig:TAPL-section-1} et
\ref{fig:TAPL-section-2} présentent les chapitres des deux premières sections\footnote{Les chapitres
en gras sont ceux qui seront formalisés.} sur lesquelles se concentrera ce projet.

%TODO Trouver pourquoi le \begin{center} ne fonctionne pas.

\begin{figure}[h]
  \begin{center}
    \begin{enumerate}[label=§ \arabic*]
        \setcounter{enumi}{2}
      \item \textbf{Expressions arithmétiques non-typées}
      \item Une implémentation en ML des expressions arithmétiques
      \item \textbf{Le lambda-calcul non-typé}
      \item Représentation non-nommé des termes
      \item Une implémentation en ML du lambda-calcul
    \end{enumerate}
  \end{center}
  \caption{Section I du livre de référence --- Les systèmes non-typés}
  \label{fig:TAPL-section-1}
\end{figure}

\begin{figure}[h]
  \begin{center}
    \begin{enumerate}[label=§ \arabic*]
        \setcounter{enumi}{7}
      \item \textbf{Expressions arithmétiques typées}
      \item \textbf{Le lambda-calcul simplement typé}
      \item Une implémentation en ML des types simples
      \item Extensions simples
      \item Normalisation
      \item Références
      \item Exceptions
    \end{enumerate}
  \end{center}
  \caption{Section II du livre de référence --- Les types simples}
  \label{fig:TAPL-section-2}
\end{figure}

Le projet formalisera donc séquentiellement les différents chapitres en se basant, au besoin, sur le
travail fait pour les chapitres précédents. Chacune des formalisations se fera en quatre étapes :

\begin{enumerate}
  \item Lecture attentive du chapitre et compréhension générale des concepts énoncés;
  \item Définition dans Isabelle/HOL des structures nécessaires à la formalisation;
  \item Preuve des différents exercices, lemmes et théorèmes avancés;
  \item Simplification des définitions et preuves.
\end{enumerate}

\section{Composition de l'équipe et rôles}

Confirmer avec monsieur Labbé que cette section n'est pas applicable.

\section{Livrables et planification}

\subsection{Description des artéfacts}

% Cette section identifie les artefacts qui seront produits durant le projet, ainsi que la
% planification de leur réalisation.

\begin{description}
  \item[Fiche de renseignement]
    Formulaire fournissant le titre du projet, un cours résumé ainsi que les noms des étudiants
    impliqués et du professeur superviseur.
  \item[Proposition de projet]
    Document présentant la problématique du projet, les objectifs, la méthodologie, les livrables,
    le plan de travail, les risques ainsi que les techniques et outils utilisés.
  \item[Rapport d'étape]
    Document présentant une version étoffée et mise à jour de la proposition de projet, ainsi qu'une
    version partielle du rapport technique.
  \item[Diapositives de la présentation orale]
    Diapositives utilisées pour la présentation orale finale du projet.
  \item[Rapport final]
    Document présentant la problématique du projet, les objectifs, la méthodologie employée, les
    hypothèses, les résultats, l'analyse des résultats, les conclusions, les recommandations et les
    références.
  \item[Théories Isabelle/HOL]
    Fichier source de l'assistant de preuve Isabelle/HOL contenant les définitions et théorèmes
    formalisés au cours de ce projet.
\end{description}

\subsection{Planification}

% Voir Annexe A. Commentez le tableau de l’annexe A. Une approche itérative est recommandée.

\section{Risques}

% Vous devez énumérer les difficultés probables que vous pourriez rencontrer et qui pourraient avoir
% un impact sur la réalisation de votre projet. Vous devez expliquer dans un tableau chaque risque
% identifié, son impact ainsi que les moyens que vous allez mettre en œuvre pour le gérer et
% atténuer sa probabilité ou son impact.

% La forme d'un risque devrait être négative. C'est-à-dire, un risque est un événement que l'on veut
% éviter. Par exemple "expérience du client" n'est pas un risque tandis que "manque d’expérience du
% client" en serait un. Voici un exemple de risque : "Un passager est blessé dans une voiture lors
% d'une collision." Une mitigation à un risque est la stratégie qui vise à éviter que l'événement
% négatif se produise. Par exemple : "Le passager porte une ceinture de sécurité et la voiture est
% équipée de coussin de sécurité gonflable." Soyez spécifique à votre projet. "Manque de temps pour
% finir le projet" et "Portée trop ambitieuse" sont des risques pour n'importe quel projet. Il n’est
% pas mauvais d’avoir des risques génériques, mais il est important d’aussi trouver plusieurs
% risques spécifiques à votre projet, de même que de trouver des mitigations qui sont spécifiques à
% votre projet.

% Si vous manquez d'inspiration, vous pouvez consulter la liste LOG_GTI_792_Generic Software Risk
% Factors.xls, disponible sur le site Web du cours sous la rubrique ‘Gabarits et guides’. Toutefois,
% faire attention de ne pas ainsi identifier des risques et mitigations qui sont applicables à la
% majorité des projets.

\begin{table}[h]
  \caption{Risques et mitigations}
  \begin{tabular}{|p{0.29\textwidth}|p{0.30\textwidth}|p{0.12\textwidth}|p{0.29\textwidth}|}
    \hline
    \textbf{Risque} &
    \textbf{Impact} &
    \textbf{Proba-bilité} &
    \textbf{Mitigation / \newline atténuation} \\
    \hline
    Manque d'expérience avec la théorie des types &
    Monopolise du temps pour apprendre la théorie &
    Faible &
    Étudier attentivement l'ouvrage de référence. \\
    \hline
    Manque d'expérience avec l'assistant de preuve Isabelle/HOL &
    Monopolise du temps pour apprendre le fonctionnement de l'outil &
    Forte &
    Étudier attentivement et se référer au besoin à la la documentation de l'outil. \\
    \hline
    Manque d'expérience en formalisation &
    Monopolise du temp pour apprendre la méthodologie &
    Forte &
    S'informer auprès des chercheurs expérimentés de la chaire de recherche. \\
    \hline
    Objectifs trop ambitieux &
    Ne pas réaliser toutes les formalisations prévues &
    Moyenne &
    Évaluer des objectifs réalistes avec mon superviseur de stage. \\
    \hline
  \end{tabular}
\end{table}

\section{Techniques et outils}

\begin{description}
  \item[Isabelle] Système générique pour l'implémentation de formalismes logiques.
  \item[Isabelle/HOL] Spécialisation d'Isabelle pour la logique d'ordre supérieur (\gls{HOL}).
  \item[Isabelle/Isar] Langage structuré permettant d'écrire des preuves plus lisibles.
  \item[Isabelle/jEdit] Environement de développement intégré pour Isabelle basé sur jEdit.
  \item[Sledgehammer] Outil appliquant des prouveurs automatiques de théorèmes ainsi que des
    solveurs de « satisfaisabilité modulo théories »
\end{description}

\section{Références}

% Vos références initiales. Par exemple: documents de projet existants, compilation de patrons,
% documentation sur la technologie, etc.

\begin{itemize}
  \item Benjamin C. Pierce, Types and Programming Languages
  \item Jasmin Christian Blanchette, Hammering Away : A User's Guide to Sledgehammer for Isabele/HOL
  \item Tobias Nipkow, Programming and Proving in Isabelle/HOL
  \item Tobias Nipkow, Lawrence C. Paulson, Markus Wenzel, Isabelle/HOL : A Proof Assistant for
    Higher-Order Logic
\end{itemize}

\section{Annexe A : Plan de travail}

% Le tableau suivant présente la planification pour la réalisation des tâches ou artefacts décrits
% précédemment.
%
% Note : il faut compléter la liste des tâches et des jalons. Assurez-vous que vos tâches sont
% exprimées en termes d'activités d'ingénierie, par exemple analyse, architecture, conception,
% implémentation, etc, et non en termes de production d'artéfacts (ex: rédaction document X). Les
% artéfacts affectés par une activité d'ingénierie sont indiqués dans la colonne
% "Livrables/Artéfacts". Si vous utilisez une approche itérative, rendez vos itérations visibles
% dans ce plan. Si le projet est réalisé en équipe, ajoutez une colonne d'effort pour chaque membre
% de l'équipe.

\end{document}
