\documentclass[a4paper, oneside, 12pt, titlepage]{article}

\usepackage[utf8]{inputenc}
\usepackage[T1]{fontenc}

\begin{document}

\title{Formalizing \emph{Types and Programming Languages} in Isabelle/HOL}
\author{Martin Desharnais}
\maketitle

\begin{abstract}
We formalized, using Isabelle/HOL, some languages presented in the first two sections, namely
"Untyped Systems" and "Simple Types", of the book \emph{Types and Programming Languages} by
Benjamen~C.~Pierce. We first begin with a short tour of lambda-calculus, type theory and the
Isabelle/HOL theorem prover before attacking the formalization per se. Starting with an arithmetic
expression language offering booleans and natural numbers, we pursue, after a brief digression to
the "de Bruijn indices", to the untyped lambda-calculus. We then return to a typed variant of the
arithmetic expression language before to conclude with the simply typed lambda-calculus.
\end{abstract}

\tableofcontents
\newpage

\section{Introduction}

This bachelor thesis deals with the formalisation, in the Isabelle/HOL theorem prover, of part of
the book \emph{Types and Programming Languages}, hereafter abbreviated \emph{TAPL}, by
Benjamen~C.~Pierce. This work concentrate on four of the languages, ranging from simple arithmetic
expressions to fully fledged lambda-calculus, presented in the first two sections, namely "Untyped
Systems" and "Simple Types".

The main motivation to have chosen this subject is the intersection of personal interested and of
opportunities provided by our intership at the chair for logic and verification at TU München.
Having gradually developped an interest for programming languages in the last years, we were
eager to learn more about the foundations of type theory. Pierce's book stand out as one of the
references recommended for a deep introduction to the main elements of the field. Also, as part of
our intership, we worked on the implementation of the (Co)datatype module of the Isabelle/HOL
theorem proover. Having experienced the implementator role, we also wanted to learn about the user
role and about the process of formalization. The choise of the subject of this this thesis was thus
a logic consequence.

Before to dig into the realm of formalizations, we first introduce the necessary background (section
\ref{sec:background}) in lambda-calculus, type systems and Isabelle/HOL.

The formalizations we performed all have a direct corespondance with chapters from \emph{TAPL}
(section \ref{sec:structure-of-formalization}).

The untyped arithmetic expressions language (section \ref{sec:untyped-arith-expr}) served as a
warm-up where to experiment with the general structure of formalizations.

The formalization of the nameless representation of terms (section \ref{sec:nameless-rep-of-terms})
was not initially planned but arrised from the need to use a concreate representation for variables
in the lambda-calculus.

While the previous formalizations were either a warm-up or a representation necessity, the untyped
lambda-calculus (section \ref{sec:untyped-lambda-calculus}) is the first fully fledged programming
language we formalized.

The typed arithmetic expressions language (section \ref{sec:typed-arith-expr}) is again a warm-up,
this time to experiment with the formalization of a type system.

The simply typed lambda-calculus (section \ref{sec:simply-typed-lambda-calculus}) is the nearest to
a real programming language that we formalize.

\section{Background}
\label{sec:background}

\subsection{Lambda-Calculus}

  * Minimalist language that contains only functions.

  * term = Var x | Abs x t | App t1 t2

  * Execution = beta-reduction
    * care must be taken to avoid name clashes

  * Turing complete

  * Expressive enough to express most features from high level programming languages.
    * e.g. (Abs x t) = single argument function (arity 1)
    * e.g. (Abs x (Abs y t)) = multi-argument function (aritiy 2)
    * e.g. (Abs x t) y = let x = y in t
    * e.g. (Abs t (Abs f (Var t))) = true
    * e.g. (Abs t (Abs f (Var f))) = false
    * there exists encoding for numbers, lists, etc.
    * there exists encoding for arbitrary recursion

  * In real programming languages, those features are built-in for performance reason.

\subsection{Type Systems}

  * Syntactic method to prove the absence of certain behaviors

  * Type systems are conservatives
    * if <complicated-expression> then 42 else true will fail to typecheck
      * even if the expression will always evaluate to true at runtime
      * even if <complicated-expression> = true because a type checker does not evaluate expressions

  * Specific requirement can be provide as type annotation
    * (maybe too complicated example)
    * f : function
    * f : a -> b
    * f : List a -> List b
    * f : List a -> List a
    * f : List N a -> List N a
    * f : (xs : List N a) -> (ys : List N a) ** Permutation xs ys
    * f : (xs : List N a) -> (ys : List N a) ** Permutation xs ys ** Ordered xs
    * The more information we put in the types
      * The more invalid programs the type checker will catch
      * The hardest it become to reason about the program
      * It is a tradeoff

  * Early detection of programming errors
    * matrix\_product : Matrix n m -> Matrix m k -> Matrix n k

  * Maintenance tool
    * Change an definition an hit compile to know which part of the program it breaks.

  * Abstraction tool
    * Abstract datatype
    * Module

  * Checked documentation

  * Language safety (e.g. memory safe)
    * A safe language is completly defined by its programmer manual.
      * C contains a lot of implementation defined and undefined behaviours.

  * Efficient language generation

\subsection{Isabelle/HOL}

  * Theory files <--> Modules

  * Definitions <--> Type and function declarations

  * Theorems <--> Assert

  * Proofs are the unique thing

  * See examples

\section{Structure of the Formalization}
\label{sec:structure-of-formalization}

  * The formalization closely follows the chapter of the book.
    * One theory per chapter.
    * The only exception is the chapter 6, Nameless Representation of Terms.
      * We chose this representation because, for ITP, we need a concreate representation and can
        not just assume alpha-conversion when necessary as is done in the book.
      * Which implies that we must cover it before the untyped lambda-calculus

\section{Untyped Arithmetic Expressions}
\label{sec:untyped-arith-expr}

  * Most our definitions closely follow the ones presented in the book
    * With the exception that some information implicit in the book must be explicitly state.
      * e.g. "nv" that stand for numeric value must be a "is\_numeric\_value nv" assumption.
    * The multistep evaluation relation is defined differently, take the shape of a list.
      * With a base reflexive case (Nil)
      * And a progress case (Cons)
      * We then supply lemmas showing that our definition exibit the same properties as the one in
        the book.

  * We had to introduce an helper lemma (eval\_once\_size\_B) that was implicit in the book about the
    fact that each evaluation step reduce the size of the term.

  * The formalization is separate in two, with first only booleans and then the fully fledged
    arithmetic expression language.
    * The second part is just an explanation and does not contains any theorems.
      * We decided to reprove, or disprove, the theorems introduced in the section on booleans.
  * Lemma 3.3.3: We choose to make an induction on the structure of terms instead of the depth.

  * Lemma 3.3.4: Trivial because already provided by Isabelle/HOL modulo schematic variable
    instanciation.

\section{Nameless Representation of Terms}
\label{sec:nameless-rep-of-terms}

  * Representation of variable which free us to consider the case of variable name clashes.

  * Abs "x" (Var "x") = Abs (Var 0)
  * Abs "x" (Abs "y" (Var "x")) = Abs (Abs (Var 1))

  * We defined a single shift function for both up and down shift.
    * It use a integer as a shift size and we convert it to a natural number when need.
      * rely on the fact that "nat (int 2 - 5) = 0" instead of -3
    * It caused small difficulties in Typed Lambda-Calculus because we had to handle both case in
      our proofs even if we were interested in only one.

\section{Untyped Lambda-Calculus}
\label{sec:untyped-lambda-calculus}

  * The main point of this chapter is the presentation of the lambda-calculus and explanations of
    why it is so important as a basic representation of programming languages.

  * The fact that Church numerals, booleans, etc. represent the real numerals and booleans took me a
    while to understand.
    * The main point of the story is that, we tend describe a concept through what can be achived
      with it i.e. the operations that are available
    * Church encoding is isomorphic to their natural counterpart because we can make the exact
      operations and change freely from one representation to the other.

  * Def 5.3.1 is quite different because we chose De Bruijn indices.
    * FV : term -> var set

  * The substitution is very different because we use De Bruijn indices while the book assumes
    alpha-conversion when necessary.
    * It did not caused problems in this chapter (but wait for simply typed lambda-calculus...)

  * I was quited shocked when I realised that the given definition of the small-step evaluation
    relation does not allow the beta reduction of (%x. x) y to y.
    * I later realized that it is explained by the concept of bounded variables, free variables and
      closed terms.

  * The chapter does not contain theorems on the language per se, so we decieded to reprove, or
    disprove, the theorems introduced in the section on booleans.

\section{Typed Arithmetic Expressions}
\label{sec:typed-arith-expr}

  * The formalisation of this chapter trivially follow the book.

\section{Simply Typed Lambda-Calculus}
\label{sec:simply-typed-lambda-calculus}

  * We chose to represent the typing context as a (type list).
    * which is equivalent to ((nat * type) * set)
    * while the book use ((string * type) * set)

  * I was shocked when I realized that the pure simply typed lambda-calculus is degenerated.
    * i.e. It have to be extand with some basic datatype to be used.
    * Booleans in our case

  * Thm 9.3.5[Progress] This is the first place where we explicitly had to state that the term is
    closed

  * Lemma 9.3.6[Permutation of context] is not a theorem in our chose representation of typing
    context.
    * Fortunatly, we manage without it.

  * Lemma 9.3.7[Weakening of context] the form is quite different because of De Bruijn indices.
    * Was difficult to come with.

  * Lemma 9.3.7[preservation under substitution] the theorem (not just the form) is different.

  * We needed to add an other helper lemma, shift\_down, to prove thm 9.3.9
    * Was horrible to come with and even more horrible to prove

  * We also needed to add FV\_shift and FV\_subst helper lemmas
    * Was moderatly difficult to come with and prove

  * Thm 9.3.9[Preservation] Was very difficult to prove because, we did not had the required helper
    lemmas (see above).
    * Once the helper lemma were provided, the proof went quite smoothly.

  * For the part on type erasure, we had to define a new untyped lambda calculus because of the
    booleans.
    * While the book only provide definitions and proofs for a subset of the simply typed lambda
      calculus, our machined checked proofs required exhaustiveness.

\section{Conclusion}

\end{document}
